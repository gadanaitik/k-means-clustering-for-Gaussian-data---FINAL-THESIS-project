\textit{k}-means is probably the most widely researched topic in clustering analysis. Unsupervised learning is a deep rabbit hole with nuanced approaches that is addressed in a lot of published papers. Deep learning has been the hot topic for the past several years that researchers seem excited about. In this pursuit of higher accuracy clustering results, \textit{k}-means certainly lends itself to set up more interesting research in the field of deep learning. Machine learning paths like computer vision, natural language processing, predictive analysis and operations research all have some form of clustering analysis associated with them.
There are few ways this project sets up work for future research:
\begin{itemize}
    \item The effect of variance on the accuracy of the proposed algorithm.
    \item The validity of the proposed algorithm on high dimensional data.
    \item Cluster center initialization methods and their effect on the overall accuracy.
    \item Using an optimization technique to address the sensitivity to the initial cluster centers.
\end{itemize}

There are many other avenues one can pursue and build upon, based on this project. The hope is that the work done in this project provides motivation and material for more sophisticated approaches towards \textit{k}-means research.